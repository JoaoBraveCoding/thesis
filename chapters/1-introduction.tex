\chapter{Introduction}
\label{chap:intro}
Cryptocurrencies, of which Bitcoin is the most popular, gained a lot of exposure in recent years. Cryptocurrencies are digital assets used as a medium of exchange between users, uses strong cryptography to secure financial transactions, control the creation of additional units, and verify the transfer of assets. With their popularisation came an increase in average daily users and as a consequence of this increase, there has also been an increase in the computing and network resources spent by the systems that maintain the cryptocurrencies. For instance, at the time of this writing, Bitcoin (the most widely used cryptocurrency) has peaked at a trading value of more than 20K USD and the Bitcoin network processed more than 250K transactions per day.\footnote{Taken from \url{https://blockchain.info/charts}.} Furthermore cryptocurrencies also have lots of advantages to the users:

\begin{itemize}
    \item \textbf{Anti fraud} - cryptocurrencies are digital and cannot be counterfeited or reversed arbitrarily by the sender, as it happens with some other technologies.
    \item \textbf{Identity Theft} - traditional payment methods like credit card work on a "pull" basis where the store indicates the payment and pulls the designated amount from the account of the user. Whereas cryptocurrencies work on a “push” mechanism that allows the user to send exactly what it wants to the recipient with no further information.
    \item \textbf{Lower Fees} ~ the fees that are imposed on transactions through cryptocurrencies are much lower than the ones imposed on more traditional methods of payment.
\end{itemize}

Moreover, the Blockchain, the main technology behind cryptocurrencies has emerged as useful for a myriad of other applications, from birth, wedding, and death certificates to the monetization of music.\footnote{Taken from \url{https://blockgeeks.com/guides/blockchain-applications/}}

Cryptocurrencies usually work as follows, Users exchanging digital assets between them through transactions. Transactions are then grouped together forming a block. Next blocks are linked together forming a chain, hence the name Blockchain. By linking blocks together we are establishing a chronological order over the blocks and the transactions inside these blocks, forming a ledger.

This ledger is maintained in a distributed fashion by all the nodes in the network and any node that is willing to spend their resources is then to add blocks to the distributed ledger. These nodes are known as miners. 

The ledger is maintained by disseminating in the network information about transactions, blocks and other metadata. As the number of users and issued transaction grows, the process of disseminating information consumes a lot of resources. For instance, if we consider the cryptocurrency  Bitcoin, at the time of writing there is an average of 200K transactions processed per day. Since each transaction usually imposes an exchange of at least 3 messages then this means that for two nodes to exchange the daily amount of transactions they have to exchange 600K messages. Furthermore, this number grows even higher if we multiply it by the 10K nodes that are in the network and by the fact that each node sends a transaction to all his neighbours. Because of these reasons the process of dissemination generates a lot of duplicated messages. However, a naive reduction in the number of messages exchanged can result in some nodes not receiving, for instance, all transactions, impacting the performance and correctness of the system.

This thesis addresses the problem of the large amount of resources spent on disseminating information in cryptocurrencies, namely Bitcoin. In particular, we propose a new protocol for the dissemination of transactions.

\section{Motivation}
\label{sec:motivation}

Our objective is to lower the amount of resources spent in the dissemination process. Since there are multiple disadvantages with nodes having to process all the extra messages that receive from their neighbours, from wasted CPU cycles to unnecessary power consumption.

However, we can not simply stop disseminating information to a fraction of the nodes without any criteria. This would not only leave the cryptocurrency vulnerable to attacks, like double-spending, but it would also lower its performance. For instance, a transaction could take a very long time to reach a miner because it was sent to nodes that had few connections to miners. 

Hence we have to take advantage of already existing characteristics in these systems and study them properly in order to come up with a solution that not only lowers the resources used but also does not put at stake the current resilience of the system. In fact, in the \textit{Bitcoin} network, only a fraction of the network (currently around $10\%$) spends resources generating new blocks. Hence, our strategy consists of skewing the dissemination algorithm such that transactions reach miners faster.


%For instance in Bitcoin, in an attack known as the partition attack,  an \acrlong{as} can isolate a partition of the network to make it vulnerable to double spending (see~\cite{apostolaki2016hijacking}). Attacks such as this are increasingly more appealing due to the significant financial advantage that an attacker can gain, if successful. It is therefore of the utmost importance to the future of cryptocurrencies to design mechanisms that eliminate, or at least mitigate the possibility of such attacks without sacrificing other important aspects of cryptocurrencies.

%We are particularly interested in studying the protocols that support information dissemination in blockchain networks, namely their efficiency, as these are the backbone that allows cryptocurrencies to function properly. Even though cryptocurrencies have attracted lots of attention from the community on higher level subjects such as the consensus protocol used, few works explore the subject of information dissemination~\cite{miller2015discovering};~\cite{heilman2015eclipse};~\cite{bojja2017dandelion};~\cite{owenson2017proximity}.
%Strikingly, the few that do, show that information dissemination in cryptocurrencies is brittle and can have a severe impact on the system~\cite{miller2015discovering}.

%Another problem with the literature developed in this field is that most of these technologies are open source. Given that in an open source architeture patches are usually introduced frequently which leads to previous work quickly becoming outdated. Furthermore, the documentation of these open source projects that currently exists is either outdated or is very basic and does not properly explains how the system works. This makes it difficult for new developers/researchers when they want to understand and work with the system so that they are able to propose changes in order to improve it.

%Leveraging these techniques, we present an architecture that aims to improve the information propagation in cryptocurrency networks. More precisely, we focus on Bitcoin, but as we will see later the approach can be applied to other cryptocurrencies as well. The reason for choosing Bitcoin is that it is the most popular cryptocurrency and its core design is shared by many other cryptocurrencies, hence improvements over Bitcoin could be easily applied in other systems.

\section{Contributions}
\label{chap:contributions}
This thesis studies in detail how the latest membership and information dissemination protocols of Bitcoin work. Based on this study, we proposed an improvement over the dissemination protocol that reduces the amount of information that needs to be exchanged without compromising correctness. In detail, we provide:
%contributes with an updated description of how the Bitcoin protocol disseminates information and builds its network. It also introduces an improvement over the current version of Bitcoin and alike Blockchain technologies. In this dissertation, we offer:
\begin{itemize}
    \item A principled analysis of the limitations of the information dissemination protocol the Bitcoin network;
    \item A study on how the protocol has been implemented in the most recent Bitcoin releases;
    \item An extensible architecture that eliminates or mitigates the identified limitations and finally;
    \item A detailed evaluation of the proposed solution.
\end{itemize}
%To more properly address our problem we will look at techniques that have been proposed in distributed systems literature such as overlay networks~\cite{jesi2009secure} or node behaviour~\cite{li2006bar}.
%\begin{itemize}
%\item An amendment to the dissemination mechanism able to reduce the amount of used bandwidth  and the number of messages sent without compromising the resilience of the current system;
%\end{itemize}
%In addition, it provides an experimental evaluation of the performance of improvement based on simulations.
%Finally, it also details the implementation of a simulator where we evaluated our solutions.

\section{Results}
\label{chap:results}

This thesis produced the following results:
\begin{itemize}
  \item A specification for a new dissemination protocol;
  \item An implementation of the proposed protocol;
  \item An experimental evaluation of the resulting system.
  \item An article accepted in INFORUM 2018 - Técnicas para Reduzir a Carga na Rede no Livro-Razão da \emph{Bitcoin}, João Marçal, Miguel Matos, Luís Rodrigues
  \item Our protocol is able to save 41.5\% of the messages sent and 10.2\% of the information sent by the traditional Bitcoin protocol.
\end{itemize}

\section{Structure of the Document}
\label{chap:sotd}

The remaining of this document is organised as follows. Chapter~\ref{chap:rw} provides an introduction to Bitcoin and its systems for information dissemination and membership management. It also covers some attacks the cryptocurrency and problems related with \acrlong{p2p}. Chapter~\ref{chap:arc} describes the system model and our protocol. Chapter~\ref{chap:evaluation} presents the experimental evaluation results. Finally, Chapter~\ref{chap:conclusions} concludes the document by summarising its main points and discussing future work.


%Like any other technology that has significant financial value, cryptocurrencies and blockchain networks have been subject to a number of attacks that attempt to exploit existing vulnerabilities in current protocols to obtain illicit profit.
%For instance in Bitcoin,  in an attack known as the partition attack, an Autonomous System (AS) can isolate a partition of the network to make it vulnerable to double spending (see~\cite{apostolaki2016hijacking}).
%Attacks such as this are increasingly more appealing due to the potential significant financial advantage that an attacker can gain, if successfull.
%
%This kind of attack has the potential to result in a significant financial advantage to the attacker and hence it
%
%In this work, we are particularly interested in studying the protocols that support information dissemination in blockchain networks, because although there has been a lot of research regarding blockchain few research studies have focused on blockchain networks and how the dissemination networks work. Information dissemination networks are the networks that support the information dissemination that allow the blockchain concept of a distributed ledger to work properly. Furthermore, information dissemination networks are a key component of existing cryptocurrency systems and have been exploited as breaches for several types of attacks. We describe the current architecture and protocols of such networks, identify their known vulnerabilities, and propose mitigation techniques to these vulnerabilities. For that purpose, we also survey a number of relevant techniques, that have been proposed in different contexts, and that can be useful to harden blockchain information dissemination networks. Finally, we present an architecture that aims at eliminating and/or mitigating current vulnerabilities in these networks.

